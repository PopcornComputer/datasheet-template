\documentclass[10pt]{datasheet}

\usepackage[utf8]{inputenc}

\title{Example Part}
\author{Solder Party}
\date{April 2021}
\version{1.0}
\website{www.solder.party}

\begin{document}

\begin{titlepage}
	\maketitle
\end{titlepage}

\pagestyle{normalpage}

\tableofcontents

\clearpage

\iftotaltables
	\listoftables
	\clearpage
\fi

\iftotalfigures
	\listoffigures
	\clearpage
\fi

\section{Introduction}

\subsection{Features}
	\begin{itemize}
		\item{Classic datasheet style in LaTeX}
		\item{Write math equations as easy as $e^{i \pi}$}
		\item{Easy to use document class}
	\end{itemize}

\subsection{Applications}
	\begin{itemize}
		\item{Data sheets for electronics components}
		\item{Technical sales brochures}
		\item{Functional specifications}
	\end{itemize}

\subsection{General Description}
	The \textbf{datasheet} document class makes it easy to write great looking
	data sheets using the LaTeX typesetting system. It follows the classic style used
	by most manufacturers of electronic components.
	
	You can download the document class from
	\href{https://github.com/PetteriAimonen/latex-datasheet-template/}{latex-datasheet-template}
	GitHub repository.
	The repository includes this example datasheet as \textbf{example.tex} and
	the document class as \textbf{datasheet.cls}.
	You can build the PDF document using command \texttt{latexmk -pdf}.

\section{Pin Definitions}

\subsection{Electrical Specifications}
	All specifications are in $-40\degree C \leq T_A \leq 85\degree C$ unless otherwise noted.
	
	\begin{table}[h]
		\begin{threeparttable}
			\caption{Example Data Sheet Specifications}
			\begin{tabularx}{\textwidth}{l | c | c c c | c | X}
				\thickhline
				\textbf{Parameter} & \textbf{Symbol} & \textbf{Min.} & \textbf{Typ.} & \textbf{Max.} &
				\textbf{Unit} & \textbf{Conditions} \\
				\hline
				Page width  & $p_w$ & 20.9 & 21.0 & 21.1 & cm & \multirow{2}{*}{Standard A4 paper} \\
				Page height & $p_h$ & 29.6 & 29.7 & 29.8 & cm &  \\
				\hline
				Insulation voltage & $E_{max}$\tnote{1} & & 1 & & kV & \\
				\thickhline
			\end{tabularx}
			
			\begin{tablenotes}
				\item[1]{Based on characterization data, not tested in production.}
			\end{tablenotes}
		\end{threeparttable}
	\end{table}

\subsection{Absolute Maximum Ratings}

	\begin{table}[h]
		\caption{Absolute Maximum Ratings of Example Data Sheet}
		\begin{tabularx}{\textwidth}{l | X}
			\thickhline
			\textbf{Parameter} & \textbf{Rating} \hspace{5cm} \\
			\hline
			Daily exposure to LaTeX & 24 hours \\
			\thickhline
		\end{tabularx}
	\end{table}
	
	\textbf{Note:} Stresses above those listed under Absolute Maximum Ratings can
	cause permanent damage to the device. This is a stress rating only. Functional
	operation of the device is not implied in any conditions above those indicated
	in the Electrical Specifications section.

\clearpage

\begin{versionhistory}
  	\vhEntry{v1.0}{April 2021}{}{First release}
\end{versionhistory}

\end{document}


